\documentclass[a4paper,12pt]{article}
%%%%%~~~~~~~~~~~~~~~~~~~~~usepackage~~~~~~~~~~~~~~~~~
\usepackage{rotating}
\usepackage[top=1in, bottom=1in, left=1in, right=1in]{geometry}
\usepackage{graphicx}
\usepackage[numbers,square,sort&compress]{natbib}
\usepackage{setspace}
\usepackage[cdot,mediumqspace,]{SIunits}
\usepackage{caption}
\usepackage{subcaption}
\usepackage{mathtools}
\usepackage{authblk} % to get affil
\usepackage{float} % to get figures where wanted
\usepackage{indentfirst} % to indent the first para of every chapter


%####################### new command

\newcommand{\myemail}{ayushi.singh@mail.utoronto.ca}
\newcommand{\anita}{anita.bahmanyar@mail.utoronto.ca}
\newcommand{\carly}{c.berard@mail.utoronot.ca}

%####################### Title and etc

\begin{document}
\onehalfspacing
\title{Astronomical Spectroscopy}
\author{Ayushi Singh, Anita Bahmanyar, Carly Berard}
\affil{\small {\myemail}}%, {\anita}, {\carly}}
\affil{\small Astronomy and Physics, University of Toronto, ON}
\date{4 November, 2013}
\maketitle
%\altaffiltext{1}{{\myemail}}
%\altaffilmark{1}
%####################### Abstract

\begin{abstract}
\label{abstract}
?????? 
\end{abstract}

%####################### Introduction
\section{Introduction}
\label{sec:introduction}
???????

%####################### Equipment
\section{Equipment}
\label{sec:equipment}
???????

%####################### Data
\section{Data Summary}
\label{sec:data}

The data collected from PMT using PMT Python Module was represented in python array of counts per sample and saved in a file. These files were then later used to manipulate the data. Multiple data was collected with different time interval and number of samples using the method explained in Section \ref{sec:equipment}. For each combination number of samples and time interval, 6 or 10 data set was collected.

%~~~~~~~~~~~~~~~~~~~~~~~~~~~~~~~~~~~~ CCD ~~~~~~~~~~~~~~~~~~~~~~~~~~~~~~~~~~~~~~~~~~~~
\begin{table}[H]
\centering % used for centering table
\caption{Data from the CCD in the lab}
\tabcolsep 2.pt %\small
\footnotesize

\begin{tabular}{ccccc}% centred columns (8 columns)
\hline
\hline

Data Number  & Source & Integration Time & Number of Samples & Comments \\
& & (ms) & &\\

 
\hline
\hline
1   &   Neon            & 100    & 1000 & pick this \\
2   &   Neon            & 100    & 6    & or this\\
2   &   Table Lamp      & 50     & 10   &\\
2   &   Table Lamp      & 100    & 1000 &\\
2   &   Table Lamp      & 200    & 10   &\\
2   &   Table Lamp      & 1000   & 10   &\\
3   &   Mercury Light   & 100    & 6    &\\
4   &   Sun             & 100    & 6    &\\
5   &   Dark            & 50     & 10   &\\
5   &   Dark            & 100    & 10   &\\
5   &   Dark            & 200    & 10   &\\
5   &   Dark            & 1000   & 10   &\\

\hline
\hline

\end{tabular}
\label{table:ccd} % is used to refer this table in the text
\end{table}
%~~~~~~~~~~~~~~~~~~~~~~~~~~~~~~~~~~~~~~~~~~~~~~~~~~~~~~~~~~~~~~~~~~~~~~~~~~~~~~~~~~~~~~~~~~

For all the data took from telescope, corresponding dark count data is was also collected. This was then used to reduce noise from the Data.
%~~~~~~~~~~~~~~~~~~~~~~~~~~~~~~~~~~~~ Telescope ~~~~~~~~~~~~~~~~~~~~~~~~~~~~~~~~~~~~~~~~~~~~
\begin{table}[H]
\centering % used for centering table
\caption{Data from the campus Telescope}
\tabcolsep 2.pt %\small
\footnotesize

\begin{tabular}{ccccc}% centred columns (8 columns)
\hline
\hline

Data Number & Source & Integration Time & Comments \\
& & (ms) &\\

 
\hline
\hline
1   &   Vega    &   & \\
2   &   Vega    &   & \\
3   &   Vega    &   & \\
4   &   Enif    &   &\\
5   &   Enif    &       &\\
6   &   Albero1 &       & one star from the binary system\\   
7   &   Albero2 &       & second star from the binary system\\
8   &   Halogen &       & used to correct for flatfield \\
9   &   Neon    &       & give us pixel to wavelength relationship\\


\hline
\hline

\end{tabular}
\label{table:telescope} % is used to refer this table in the text
\end{table}
%~~~~~~~~~~~~~~~~~~~~~~~~~~~~~~~~~~~~~~~~~~~~~~~~~~~~~~~~~~~~~~~~~~~~~~~~~~~~~~~~~~~~~~~~~~

%####################### Data Reduction and Method
\section{Data Reduction and Method}
\label{sec:reduction}

%####################### Calculation and Modeling
\section{Calculation and Modeling}
\label{sec:calc}

%########## Lab CCD based calcuation

\subsection{Noise and Gain in CCD data} 
\label{sec:noise_ccd}

Taking the mean and variance of each pixel in all all of the data using equations

%~~~~~~~~~~~~~~~~~~~~~~~~~~~~~~~~~~~~
\begin{equation}
\label{eq:mean}
\bar{x} = \frac{\sum x_i}{N},
\end{equation}
where $\bar{x}$ is the mean and $N$ is number of samples. This is the average value of the data $x$ \cite{error}.
%~~~~~~~~~~~~~~~~~~~~~~~~~~~~~~~~~~~~
\begin{equation}
\label{eq:variance}
s^2_{ADU} = \frac{\sum x_i - \bar{x}}{N-1},
\end{equation}
where $s^2_{ADU}$ is the variance. This value represents the standard deviation (uncertainty in the measurement) squared for that data set \cite{error}. 
%~~~~~~~~~~~~~~~~~~~~~~~~~~~~~~~~~~~~

%##########Telescope based calcuation 
\subsection{Pixel calibration in Telescope Data} 
\label{sec:noise_tele}

Flatfield...$P_i$ is the calibrated intensity for each pixel using following equations \cite{instructions} 

%~~~~~~~~~~~~~~~~~~~~~~~~~~~~~~~~~~~~
\begin{equation}
\label{gain}
s^2_{ADU} = s^2_0 + k\bar{x}_{ADU},
\end{equation}
$s^2_0$ values give us the read noice and $k$ is $\frac{1}{gain}$. 

%~~~~~~~~~~~~~~~~~~~~~~~~~~~~~~~~~~~~
\begin{equation}
\label{blackb}
P_i = {\frac{R_i-D_{Ri}}{L_i-D_{Li}}B(\nu_i,T)},
\end{equation}
where $R_i$ is the raw signal, $L_i$ is the Lamp, $D_{Ri}$ and $D_{Li}$ is the dark count for raw signal and halogen lamp, respectivaly. $B_{\nu}(T)$ is the Planck fucntion.  
%~~~~~~~~~~~~~~~~~~~~~~~~~~~~~~~~~~~~
\begin{equation}
\label{planck}
B(\nu,T) = {\frac{2h\nu^3}{c^2}\frac{1}{exp(h\nu/kT)-1}},
\end{equation}
where $\nu_i=c/\lambda_i$, $\lambda_i$ being the corresponding wavelength for each pixel. The value for $T$ for halogen lamp is approxmitaly 3200 K.
%~~~~~~~~~~~~~~~~~~~~~~~~~~~~~~~~~~~~

%~~~~~~~~~~~~~~~~~~~~~~~~~~~~~~~ Figure ~~~~~~~~~~~~~~~~~~~~~~~~~~~~~~~~~~~~~~~~~~~~~~~
\begin{figure}[H]
\centering
\includegraphics[angle=0,height=12cm,width=15.5cm]{universe.jpg}
\caption{Time vs. count per sample graph for data number 1 from Table \ref{table:telescope}. There are six sets of data, where time interval of each sample is 0.001s for 100 samples.}
\label{fig:task1_plots}
\end{figure}
%~~~~~~~~~~~~~~~~~~~~~~~~~~~~~~~~~~~~~~~~~~~~~~~~~~~~~~~~~~~~~~~~~~~~~~~~~~~~~~~~~~~~~~~~

%####################### Discussion 
\section{Discussion}
\label{sec:discuss}

%####################### CCD
\subsection{CCD} 
\label{sec:ccd}

%####################### Telescope
\subsection{Telescope} 
\label{sec:telescope}

%######################### Conclusion
\section{Conclusion}
??????

%\bibliographystyle{plain}
%\bibliography{references.bib}
\end{document}
